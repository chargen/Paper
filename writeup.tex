\documentclass[10pt,twocolumn]{article}
\usepackage{fullpage}
\usepackage{times}
\usepackage{amsmath,proof,amsthm,amssymb,color}
%\usepackage{multicol}
\usepackage{float}
\usepackage{graphicx}
\floatstyle{boxed} 
\restylefloat{figure}
\usepackage{hyperref}
\usepackage{listings}
%\usepackage[all]{hypcap}


\title{virtexsquared: ARM-like System-on-Chip on an FPGA}

\author{Joshua Wise\\
\texttt{$<$jwise@andrew.cmu.edu$>$} \and
Josiah Boning\\
\texttt{$<$jboning@andrew.cmu.edu$>$} \and
Bradley Yoo\\
\texttt{$<$bjy@andrew.cmu.edu$>$}}


\begin{document}
\maketitle

\begin{abstract}

The authors provide a report on the completed development of an ARM-like
System-on-Chip built on a Xilinx Virtex-5 FPGA.  A high-level overview of
the design is provided, as well as detailed examinations of various
submodules within the system.  

\end{abstract}

\vspace{0.1in} % fff fu LaTeX

\section{Introduction}

In fulfillment of the 18-545 Advanced Digital Design Capstone, this group
designed and implemented a peripheral layer for an ARM-like core. 
The system is built around two memory buses, and contains I/O cores to
interface with a smattering of peripherals, including:

\begin{itemize}
\item{DVI/VGA output from Chrontel CH7301C; video output is cached in a
framebuffer in main memory.}
\item{RS-232 serial output.}
\item{AC'97 audio with Analog AD1981B; audio output is cached in a
a simple main memory buffer.}
\item{PS/2 keyboard.}
\item{CompactFlash via the Xilinx SystemACE controller.}
\item{DDR2 SODIMM; interface glue is provided using Xilinx's predesigned
Memory Interface Generator (MIG) IP.}
\end{itemize}

The system contains various non-I/O cores to provide additional functionality,
including:

\begin{itemize}
\item{Timer}
\item{Bootstrap program preloader}
\item{Accelerators: memory set and image clear}
\end{itemize}

A primary goal of this peripheral layer is to be reused in 18-447; the system
is designed to be simple to interface with from a core perspective, and
hence usable for students in an introductory computer architecture class. 
The core, in particular, has a moderate \textit{anti}-goal of high
performance.

\section{Memory Architecture}

The system, as a whole, is partitioned into two major access paths -- the
FSAB (\textit{Fast System Access Bus}), and the SPAM Bus (\textit{Slow
Peripheral Access Memory} Bus).  These two buses were designed to meet
wildly differing goals; the FSAB is designed for high-bandwidth, cachable
transactions, whereas the SPAMBus is designed for configuration space
register (CSR) accesses.  Ideally, most of the system's traffic will occur
over the FSAB.

\begin{figure}
  \centering
    %\includegraphics[width=0.4\textwidth]{Tile-diagram.pdf}
    Eat it.
  \caption{High-level system diagram.} \label{system_diagram}
\end{figure}

\subsection{Fast System Access Bus}

The FSAB is a transaction-oriented bus designed primarily for accessing high-
latency memories such as DRAMs. The vast majority of the memory accesses on 
the system during normal operation (i.e., not at boot/ configuration time) 
will happen via the FSAB. 

\subsubsection{Terminology}

The FSAB works in terms of `transactions'. A read transaction shall consist 
of the `read request', and `read data' phases. A write transaction shall 
consist only of the `write data' phase.

Devices attached to the FSAB are classified as `masters' and `slaves'; a 
device that initiates transactions (reads and writes to main memory) is a 
master, and a device that completes transactions is a slave.

At times, it may be useful to discuss directions on the FSAB. For the purposes
of this discussion, data that is traveling from a master to a slave is 
considered `outbound' (and so signals for that purpose will begin with fsabo); 
similarly, data that initiates at the slave and returns to a master is 
considered `inbound' (and so signals for that purpose will begin with fsabi).


\subsubsection{Conceptual Overview}

virtexsquared, in its first incarnation, will only have a single memory 
controller. For that reason, the FSAB will be defined to have only one slave 
device - but since many peripherals may wish to do DMA access to or from main 
memory, the FSAB will be defined to potentially have many masters. The fact 
that only one slave device exists shall not be extensively exploited in either 
the design or implementation of the FSAB, since at a later time, a second 
memory controller may be interesting.

Since many memory controllers are capable of handling multiple pipelined 
requests, the FSAB supports multiple transactions in flight at a time. The 
slave will arbitrate these requests with a debit-credit system. Similarly, 
contention between masters will be resolved with an arbiter module that also 
performs queueing and debit-credit arbitration. In this regard, the bus 
arbiter should be ``invisible'' to a master; the debit-credit system should be 
interface-identical as if the master were talking directly to the slave.

Most operations that peripherals on the FSAB will perform will be in similar 
sizes. For instance, the I\$ and D\$ will always read sizes of a single cache 
line; the TLB will generally read page directory entries at a time; and the 
framebuffer will generally load half a FIFO's worth or so at a time. The 
general case, anyway, will not be a read of a single word. Similarly, writes 
will often be localized to each other. To facilitate this, each FSAB 
transaction will have a count of words up to a defined maximum of 8 along with 
the command and address. Inside each transaction, the address being written to 
shall autoincrement with each datum sent. Since not every datum may be 
interesting for a write, a ``write mask'' is also sent with every word written.

Since masters are isolated from each other by one or more arbiters, it is 
generally not possible for one master to know what another master is doing.
\footnote{This combined with the fact that writes happen on a delay -- that
is, when the memory controller gets around to it -- may pose a problem. 
Specifically, a program that writes to memory and then causes the cache line
to be read in again (or that kicks off a read on another device using the
SPAM) should be careful to verify that the write completes before the read
completes.  Depending on the specifics of the design and the ordering, this
may or may not be a problem, but it should be carefully considered.} Lacking 
any other form of cache coherency protocols (such as MOESI), a system based 
around the FSAB is not cache-coherent! This means that the applications 
programmer must take extra care to make sure that all data is synchronized to 
main memory before allowing another peripheral to access said data. (For 
instance, if the programmer wishes to write new data to the frame buffer 
before flipping pages, he must cause the CPU to clean that cache line by some 
mechanism first. In the current implementation, the data cache is implemented 
as being write-through, so no such synchronization issue takes place.)

The bus protocol is designed to be easy to implement by a DRAM-based slave. 
For that reason, the length of a read or write request is limited not just by 
the maximum, but by the alignment of the request. That is to say, if a request 
is only aligned to a four-qword boundary, but not to an eight-qword boundary, 
then the maximum size of the request is four qwords; a request of five or more 
qwords results in undefined behavior. (On the current DRAM-backed 
implementation, the resulting data comes from wrapping around the row buffer; 
on the current simulation-backed implementation, the resulting data comes from 
subsequent linear addresses.) 

\subsubsection{Design Overview: Portlist}

A FSAB master has the following ports outbound:

\begin{lstlisting}[basicstyle=\footnotesize,language=Verilog]
output                  fsabo_valid;
output [FSAB_REQ_HI:0]  fsabo_mode;
output [FSAB_DID_HI:0]  fsabo_did;
output [FSAB_DID_HI:0]  fsabo_subdid;
output [FSAB_ADDR_HI:0] fsabo_addr;
output [FSAB_LEN_HI:0]  fsabo_len;
output [FSAB_DATA_HI:0] fsabo_data;
output [FSAB_MASK_HI:0] fsabo_mask;
input                   fsabo_credit;
\end{lstlisting}

A FSAB master has the following ports inbound:

\begin{lstlisting}[basicstyle=\footnotesize,language=Verilog]
input                   fsabi_valid;
input  [FSAB_DID_HI:0]  fsabi_did;
input  [FSAB_DID_HI:0]  fsabi_subdid;
input  [FSAB_DATA_HI:0] fsabi_data;
\end{lstlisting}

No debit output (or input) is needed; a debit occurs implicitly when a new
FSAB transaction is begun. The credit input is always considered valid, even
if the inbound valid flag is not set.

\subsubsection{Design Overview: Transaction Specifics}

Transactions are defined by a start packet (and potentially additional data
packets) being sent to a slave, and the slave responding with any necessary
data packets. The slave may pipeline transactions as much as it is able
(credits can be returned before the slave has responded), but the slave
shall never reorder transactions. (Doing so would result in indeterminate
behavior with reads after writes, among other things.)

A start packet shall consist of the valid bit being set, the mode flag being
set to either FSAB\_READ or FSAB\_WRITE, the did/subdid set to the appropriate
values, an address, a length, and the first word of data and mask, should
they be valid for the current operation. Once a transaction is being sent
outbound, it shall not be interrupted by another transaction; the next data
packets are always part of this transaction. (For this reason, a master
should attempt to send packets as quickly as possible; excessive delay
between subsequent assertion of the valid flag may result in poor bus
performance.) In subsequent data packets, all control flags (i.e., all but
valid, data, and mask) are ignored.

The debit/credit system shall count each transaction as a debit, not each
word. Another transaction may be sent immediately after the credit flag is
asserted (or the credit may be queued).

The did field of each transaction shall be set as per each master's DID
assignment. The subdid may be set according to any internal state that the
master needs to track.

Some attention should be given to the mechanisms of clock distribution. In 
particular, it is permissible (and indeed often the case) that the inbound FSAB 
and outbound FSAB buses may be on different clock domains; usually the inbound 
FSAB clock comes directly from the slave, but the outbound FSAB clock is 
synchronous with the logic from the driving module. The outbound FSAB interface 
is then synchronized to the slave's clock in the arbiter. \footnote{This may be
considered a bug, as it departs from the conceptual goal that the bus arbiter 
is invisible to the master. From this perspective, it is preferable to have all
portions of the FSAB on the same clock domain, i.e. the slave's.}


\subsubsection{Known implementations}

The following are known implementations of the FSAB specification:

\begin{itemize}
\item{Caches}
\item{Memory Controller}
\item{Arbiter}
\item{DMA Controller}
\end{itemize}

\subsection{Slow Peripheral Access Memory}

The SPAM bus is a word-oriented one-access-at-a-time bus designed primarily
for accessing configuration space registers (CSRs) on peripherals.

\subsubsection{Conceptual Overview}

In theory, all peripherals on the system will be matching and decoding on
the SPAM bus; when the processor does a SPAM-bus access, it is almost
certainly the case that a device will respond.  So, the bus should be
optimized for the common case in which a device responds shortly after a
request is sent.  Similarly, it should be the case that only one device ever
matches, so the bus need not be arbitrated between them; a simple OR'ing of
responses should work.

No peripheral should need to access any other peripheral's CSRs; any such
accesses from a debug interface should be multiplexed in between the CPU and
the SPAM-bus.  As such, the master-slave relationship in the SPAM-bus is
exactly the opposite of what it is on the FSAB; there is only one SPAM-bus
master, and there can be many SPAM-bus slaves.  Again, this property should
not be extensively exploited, but it is the case in the current
implementation.

Requests on the SPAM-bus should ordinarily be low latency, but there may be
need for wait states for various reasons.  For instance, if the other
peripheral is far away on the die, it may take an extra clock to compute the
appropriate response for a read.  A write, perhaps, may block until some
(short) action is complete.  There may also be a need to block because the
target peripheral is in a different clock domain, and a synchronizer needs
to shift the data over.

If no peripheral answers by deasserting busy\_b within a reasonable number
of cycles (usually 256), the request is deemed to have timed out, and a read
should return a value indicating that (usually something along the lines of
0xDEADDEAD).

\subsubsection{Design Overview: Portlist}

The processor has the following ports outbound:

\begin{lstlisting}[basicstyle=\footnotesize,language=Verilog]
output                  spamo_valid;
output                  spamo_r_nw;
output [SPAM_DID_HI:0]  spamo_did;
output [SPAM_ADDR_HI:0] spamo_addr;
output [SPAM_DATA_HI:0] spamo_data;
\end{lstlisting}

The processor has the following ports inbound:

\begin{lstlisting}[basicstyle=\footnotesize,language=Verilog]
input                   cio__spami_busy_b;
input  [SPAM_DATA_HI:0] cio__spami_data;
\end{lstlisting}

\subsubsection{Design Overview: Lifecycle of a request}

A request begins by the processor asserting the valid flag and specifying
the rest of the fields. This happens for exactly one cycle. Some time later
(potentially even on the same cycle), a device may assert the busy\_b flag
for one cycle to indicate that the request has been filled. Inbound data
shall be valid on the same cycle.

No new request shall be specified until either a previous request has returned, 
or a previous request has timed out. A request shall time out after no fewer 
than 128 cycles; the current DCache implementation times out after 256 cycles. 
When a device is not specifying data inbound, it shall drive its output port to 
zero; this allows the bus to simply OR together all responses.

Since the SPAM bus is always on the core's clock domain, some synchronization 
may be needed for CSRs on a different clock domain. Standard modules have been 
provided to synchronize reads and writes; for more information on those, see 
the documentation for the CSRAsync modules. 

\section{Core Microarchitecture}

\section{Peripheral Architecture}

\section{Tools}

\end{document}
